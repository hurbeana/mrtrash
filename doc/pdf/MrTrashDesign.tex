\documentclass[a4paper]{scrreprt}

%% Language and font encodings
\usepackage[ngerman]{babel}
\usepackage[utf8x]{inputenc}
\usepackage[T1]{fontenc}

%% Sets page size and margins
\usepackage[a4paper,top=2.5cm,bottom=2cm,left=2cm,right=2cm]{geometry}

%% Useful packages
\usepackage{amsmath}
\usepackage{graphicx}
\usepackage[colorinlistoftodos]{todonotes}
\usepackage[colorlinks=true, allcolors=blue]{hyperref}

\title{Mr. Trash}
\subtitle{"YOUR GAME IN ONE LINE" (Witcher e.g.: "Skyrim, but with Story of Game of Throne")}
\author{
    Hurbean, Alexander\\
    \texttt{e1625747}
    \and
    Schweiger, Phillipp\\
    \texttt{e01634254}}
\titlehead{\centering\includegraphics[width=6cm]{test}}


\begin{document}
\maketitle

\null\vfill
\noindent
App Design Document\\ 
Mobile (App) Software Engineering\\
Technische Universität Wien\\
Version v1.0, März 2019\\
\newpage

%\begin{abstract}
%Short abstract of the game (max 150 words) 
%\end{abstract}

\tableofcontents

% ______________________
% chapter Overview
% ______________________
\chapter{Konzept}

Main features and aspects of your game on a first page, describing story elements. -> "selling page", publisher should be able to decide after reading this single page whether to buy in or not 

% ______________________
% chapter References
% ______________________

\chapter{Graphischer Entwurf} 

\section{High-Fidelity Mockup}
\begin{figure}
\centering
\includegraphics[width=0.3\textwidth]{test.jpg}
\caption{\label{fig:art1} Art example}
\end{figure}

\section{Product Icon}
\begin{figure}
\centering
\includegraphics[width=0.3\textwidth]{test.jpg}
\caption{\label{fig:art2} Art example}
\end{figure}

\bibliographystyle{alpha}
\bibliography{sample}

\end{document}